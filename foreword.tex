% foreword.tex
% Graphs in Computer Science (in Russian)
% Author: Evgeny Simonenko <easimonenko@mail.ru>
% License: CC BY-ND 4.0

\section*{Предисловие}
\addcontentsline{toc}{section}{Предисловие}

Теория графов – один из самых молодых разделов дискретной математики. 
Возникновение теории графов обязано работе Эйлера\footnote{Эти строки я пишу 
всего в паре кварталов по набережной Лейтенанта Шмидта от дома на Неве в 
Санкт-Петербурге, где жил Леонард Эйлер, великий российский и германский 
математик.}, датированной 1736 годом, опубликованной в 1741 году и посвящённой 
решению задачи о Кёнигсбергских мостах (смотри \cite{Fleischner}). Первое 
комплексное изложение теории графов было сделано венгерским математиком Денешом 
Кёнигом в книге <<Theorie der endlichen und unendlichen Graphen>> (Теория 
конечных и бесконечных графов) \cite{Koenig}, изданной в 1936 году. Первой 
книгой по теории графов на русском языке стал перевод с французского книги Берж 
<<Теория графов и её применения>> \cite{Berge}, выпущенный в 1962 году. Второй, 
также переводной книгой, стала книга норвежского математика Ойстина Оре 
<<Теория графов>> (1968) \cite{Ore}. С тех пор было издано огромное число книг 
как переводных, так и написанных советскими и российскими математиками, здесь 
отмечу, например, \cite{Okulov} и \cite{Lando}.

В этой книге, <<Графы в компьютерных науках>>, рассматриваются различные 
аспекты обработки графов на компьютерах и применения их в компьютерных науках.
Главное отличие этой книги о графах в том, что она не является пространным 
изложением теории графов или её отдельных аспектов. В современной теории 
графов, да и вообще в теории графов, много интересных задач и результатов, 
некоторые из которых можно узнать из современных учебников по теории графов или 
дискретной математике. Однако на момент написания этой книги автор не обнаружил 
легко доступного обобщения и систематического изложения вопросов работы с 
графами на компьютере и их применения в компьютерных науках, что и сподвигло к 
написанию этой книги параллельно занятиям этой тематикой. В главе <<Основные 
понятия теории графов>> даются все необходимые базовые определения и 
результаты, благодаря которым книгу можно читать без предварительного изучения 
теории графов.

В репозитории \url{https://github.com/easimonenko/graphs-in-computer-science} 
находятся полные исходные тексты книги в \LaTeX, а также сборка в формате PDF 
для печати на бумаге формата A4. Вы также можете адаптировать эту книгу для 
себя для другого подходящего вам формата. Эта книга распространяется под 
лицензией
\href{https://creativecommons.org/licenses/by-nd/4.0/legalcode}{CC BY-ND 4.0}. 
Если вы хотите внести изменения и распространять изменённый вариант этой книги, 
вам нужно обратиться к автору за получением отдельного разрешения. Вы можете 
произвести исправления и дополнения к этой книге и передать их автору этой 
книги для рассмотрения им возможного внесения этих изменений в исходный вариант 
при сохранении лицензии на эту книгу.
